\subsection{Gestione Organizzativa}
\subsubsection{Scopo}
Lo scopo di questa parte è quello di fornire un insieme organizzato di attività al fine di:
\begin{itemize}
    \item Fare proprio un modello organizzativo per il tracciamento dei rischi.
    \item Stabilire ruoli per pianificare il lavoro e rispettare le scadenze.
    \item Scegliere gli strumenti per le comunicazioni interne e esterne.
    \item Decidere un modello di sviluppo.
\end{itemize}

\subsubsection{Aspettative}
Le aspettative per questa parte sono:
\begin{itemize}
    \item Ottenere un piano di schemi da seguire.
    \item Definire i ruoli all'interno del gruppo.
    \item Agevolare le comunicazioni interne e esterne al gruppo.
    \item Controllare il progetto e le attività del gruppo.
\end{itemize}

\subsubsection{Descrizione}
Di seguito vengono riportati gli argomenti delle attività di gestione:
\begin{itemize}
    \item Definizione dei ruoli e dei compiti assegnati ai componenti del gruppo.
    \item Gestione delle comunicazioni.
    \item Modalità di esecuzione delle attività.
    \item Esame dei progressi delle attività.
    \item Stima dei tempi, risorse e costi.
\end{itemize}

\subsubsection{Ruoli di progetto}
I ruoli del progetto saranno ricoperti da ogni membro del gruppo in rotazione per permettere
una equa distribuzione delle mansioni da svolgere. L'organizzazione delle attività
è esposta nel \emph{Piano di Progetto} e deve essere seguita dalle varie figure progettuali, ovvero:

\paragraph{Responsabile.}
Il responsabile è quella figura che si occupa della parte di coordinamento, di pianificazione del progetto 
e di mediazione con i soggetti esterni al gruppo. Si occupa di:
\begin{itemize}
    \item Controllare le attività del team.
    \item Coordinare i membri del gruppo.
    \item Approvare i documenti.
    \item Gestire le relazioni esterne.
\end{itemize}

\paragraph{Amministratore.}
L'amministratore è colui che gestisce l'ambiente di lavoro all'interno del gruppo e le attività che svolge sono:
\begin{itemize}
    \item Controllare le infrastrutture di supporto.
    \item Documentare le regole e gli strumenti utilizzati.
    \item Attuare le scelte tecnologiche fissate dal gruppo.
    \item Controllare le configurazioni e le versioni.
\end{itemize}

\paragraph{Analista.}
L'analista ha il dovere di identificare e comprendere il dominio del problema per consentire, successivamente,
una corretta progettazione. I suoi compiti sono:
\begin{itemize}
    \item Analizzare il dominio del problema.
    \item Scrivere l'\emph{Analisi dei Requisiti}.
\end{itemize}
L'analista partecipa al progetto fino a quando non si conclude l'analisi del problema.

\paragraph{Progettista.}
Il progettista concorre alla ricerca di una soluzione per il prodotto e alle scelte tecniche e tecnologiche.
Partecipa allo sviluppo software ma non alla manutenzione. Le sue attività sono:
\begin{itemize}
    \item Progettare l'architettura dell'applicativo in modo che sia mantenibile e affidabile.
    \item Trovare soluzioni efficienti ai problemi tecnici e tecnologici del progetto.
    \item Controllare la fase di sviluppo.
\end{itemize}

\paragraph{Programmatore.}
Il programmatore copre la parte di codifica del progetto usando le soluzioni e tecnologie stabilite dal team, inoltre
spetta a questa figura la scrittura dei test per la validazione. Ricapitolando si occupa di:
\begin{itemize}
    \item Scrivere il codice che implementi le soluzioni trovate dal progettista.
    \item Realizzare i test per la verifica e validazione del software.
\end{itemize}

\paragraph{Verificatore.}
Il verificatore è tenuto a esaminare i progressi del lavoro compiuto dagli altri membri del gruppo.   
L'attività di verifica viene condotta sul codice e sui documenti col fine di far rispettare le \emph{Norme di Progetto}.

\subsubsection{Gestione delle comunicazioni}
\paragraph{Comunicazioni interne.}
Per le comunicazioni interne il gruppo si avvale di due applicazioni, ovvero Zoom Meetings e Telegram.
Zoom è una piattaforma di videoconferenza conosciuta da tutti i membri del team e adatta per le riunioni, mentre Telegram
è una applicazione di messaggistica utile per le conversazioni più rapide.

\paragraph{Comunicazioni esterne.}
Le comunicazioni esterne avvengono attraverso:
\begin{itemize}
    \item Email con l'indirizzo di posta elettronica del gruppo (\href{mailto:swe7.team@gmail.com}{swe7.team@gmail.com}).
    \item Telegram per comunicazioni veloci con Imola Informatica.
    \item Zoom per le riunioni con il proponente.
\end{itemize}

\subsubsection{Gestione delle riunioni}
\paragraph{Riunioni interne.}
Le riunioni interne avvengono su Zoom come detto al punto \$4.1.5, il link per collegarsi viene inviato, generalmente, 
qualche minuto prima dell'inizio. Ad ogni riunione i membri del gruppo si aggiornano sul lavoro svolto, discutono dei problemi e/o dubbi incontrati 
e a seguire stabiliscono le attività da svolgere per il prossimo incontro. Dopo aver riempito un foglio Excel con i propri impegni, 
i componenti del team hanno scelto un giorno in cui effettuare la riunione settimanale.

\paragraph{Riunioni esterne.}
Le riunioni esterne avvengono tramite Zoom sia con il proponente che con il committente. Prima di ogni incontro con 
l'azienda Imola, vengono inviate delle mail per concordare la data e le coordinate della riunione. 

\paragraph{Verbali.}
In tutte le riunioni viene stabilito un Redattore il quale si occupa di riassumere tutto ciò che
viene detto durante il meeting e un Verificatore il quale si assicura che non vi siano errori. La struttura del verbale viene 
largamente approfondita nei processi di supporto al punto \$3.1.6.

\subsubsection{Strumenti}
Durante il progetto verranno utilizzati i seguenti strumenti:
\begin{itemize}
    \item \textbf{Telegram}, per le comunicazioni veloci all'interno del gruppo e con il proponente.
    \item \textbf{Zoom}, per riunioni interne e esterne.
    \item \textbf{GitHub}, come archivio per il codice e la documentazione.
\end{itemize}