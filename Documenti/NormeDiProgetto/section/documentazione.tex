\subsection{Documentazione}
\subsubsection{Scopo}
Lo scopo di questa sezione è di riportare tutte le regole, template e decisioni 
per la stesura di tutti i documenti del gruppo Sweven. \newline
Tutti i documenti verranno scritti in \LaTeX

\subsubsection{Ciclo di vita dei documenti}
Il documento viene innanzitutto pianificato, cioè ci si chiede perchè è necessario il documento
e si pensa al suo contento. Gli amministratori nel ruolo di redattori, creano e scrivono 
il documento, poi i verificatori lo controllano e il responsabile approva il documento 
finito e pronto ad essere ufficialmente pubblicato. \newline
Se i verificatori trovano errori superficiali e/o di ortografia correggono immediatamente,
mentre se l'errore è più profondo o è necessario riscrivere delle parti allora si
incarica nuovamente il redattore di sistemare e poi il verificatore controllerà nuovamente.

\subsubsection{Versionamento dei documenti}
I documenti sono redatti in maniera incrementale e in più momenti, quindi per una migliore 
gestione si tengono versionati, ciò aiuta anche a capire quali sezioni sono ancora da verificare.
\newline Versione: x.y.z
\begin{description}
    \item \textbf{x} è il valore più grande e indica quale versione è stata ufficialmente pubblicata.
            Viene aggiornato solo dal responsabile quando l'approvazione del documento ha esito positivo
            e si riportano a zero sia y che z
    \item \textbf{y} è il valore che indica lo stato di verifica del documento. Viene aggiornato dal verificatore 
            quando la verifica del documento ha esito positivo, e si riporta a zero z
    \item \textbf{z} è il valore che indica lo stato di stesura del documento. Viene aggiornato dall'amministratore 
            quando redige una nuova parte del documento o aggiorna una parte già scritta.
\end{description}
Quindi il documento alla sua creazione è già alla versione 0.0.1 e a fine progetto è necessario che 
tutti i documenti saranno alla versione x.0.0

\subsubsection{Suddivisione documenti interni ed esterni}
I documenti si suddividono in interni ed esterni in base allo scopo del documento e ai destinatari.
Nei documenti interni i destinatari sono i membri del gruppo Sweven, nei documenti esterni anche i 
professori committenti e l'azienda proponente. \newline
\textbf{Interni:}
\begin{itemize}
    \item \textbf{Verbali} di riunioni interne tra i componenti del gruppo Sweven
    \item \textbf{Norme di Progetto} nei destinatari si è deciso di mettere tutti per conoscenza
    \item \textbf{Glossario} nei destinatari si è deciso di mettere tutti per conoscenza
\end{itemize}

\textbf{Esterni:}
\begin{itemize}
    \item \textbf{Verbali} di riunioni con il gruppo Sweven e l'azienda proponente
    \item \textbf{Candidatura}
    \item \textbf{Analisi dei Requisiti}
    \item \textbf{Piano di Progetto}
    \item \textbf{Piano di Qualifica}
\end{itemize}

\subsubsection{Template per tutti i documenti}
In questa sezione verranno illustrate le parti comuni a tutti i documenti, 
eventuali parti aggiuntive specifiche per una tipologia di documenti
verrà spiegata nella sezione dedicata a quella tipologia. \newline
Si è deciso che tutti i documenti vengono suddivisi in varie sezioni 
e ad ognuna corrisponde un file così da rendere più agevole l'aggiornamento 
e la revisione dei file, oltre a garantire la possibilità di 
lavorare in contemporanea in maniera asincrona allo stesso file 
(è sufficiente lavorare in sezioni diverse).

\begin{enumerate}
    \item \textbf{configuration}
            Costituisce il file principale del documento, contiene tutti i comandi, 
            i pacchetti necessari e riporta le regole generali del documento come 
            i margini, lo stile della pagina, l'intestazione, la numerazione. 
            Inoltre contiene il link a tutti gli altri file che costituiscono le 
            varie parti del documento.

    \item \textbf{frontespizio}
            Il file rappresenta la prima pagina del documento, e quindi in alto si
            è lasciato un notevole spazio bianco, il contenuto è tutto centrato e 
            inizia con il nome del documento, poi l'immagine del logo con sotto 
            nome ed email del gruppo Sweven Team. Poi c'è una tabella centrata 
            con due colonne (visibile solo riga superiore e bordo centrale delle colonne), 
            in cui vengono riporte le varie informazioni, che sono state settate all'inizio 
            del file configuration:
            \begin{itemize}
                \item Versione
                \item Uso
                \item Destinatari
                \item Stato
                \item Redattori
                \item Verificatori
                \item Approvatori
            \end{itemize}
            L'elenco dei destinatari, redattori, verificatori e approvatori 
            può essere più di una persona e affinchè i nomi siano scritti uno sotto 
            l'altro, nel comando si può scrivere Nome\textbackslash\textbackslash 
            \& Nome\textbackslash\textbackslash \& Nome\textbackslash\textbackslash
            \newline Dopo altro spazio si trova la sintesi del documento, questa frase
            ha lo scopo di rappresentare molto sinteticamente il contenuto del documento 
            così da permettere al lettore di capire dal frontespizio se è di suo interesse
            o meno.
    
    \item \textbf{diario delle modifiche}
            Il diario dellle modifiche è costituito da una tabella con 5 colonne e tutti i 
            bordi anche delle righe visibili. Le 3 colonne Versione, Data, Ruolo (ultima colonna) 
            sono state impostate center mentre per le altre due Descrizione e Autore è 
            rispettivamente stata data la dimensione di 12em e 7em, inoltre il testo è allineato 
            a sinistra. 
            Viene aumentata l'ampiezza righe ad 1.8 così da non avere tutte le righe attaccate. \newline
            La nuova riga la si aggiunge sempre ad inizio tabella così da ottenere che la prima 
            riga comunica quale è stata l'ultima modifica al documento e la versione qui riportata 
            deve corrispondere a quella scritta nel frontespizio. \newline
            La data viene scritta in formato americano aaaa-mm-gg
            Nella colonna autore non si suddividono in sillabe i nomi o cognomi, se capita usare
            Nome \textbackslash newline Cognome.


            Dopo il diario delle modifiche, nel file configuration c'è il comando di creare 
            l'indice del documento.

    \item \textbf{contenuto} (possono essere anche più file)
            Le pagine successive contengono il contenuto vero e proprio del documento seguendo 
            l'indice. In tutte le pagine diverse dal frontespizio è prevista un'intestazione in 
            grigio in cui a sinistra c'è il nome del gruppo e a destra si riporta il nome del documento.
            Mentre nel piè di pagina viene riportato il numero della pagina rispetto alle pagine totali.
\end{enumerate}

\subsubsection{Verbali}
Nei verbali di diverso rispetto a quanto scritto sopra nella sezione 3.1.5 
sono presenti altri due file di template: "informazioni" e "conclusioni-decisioni", 
quindi l'ordine delle varie parti sarà il seguente:
\begin{enumerate}
        \item \textbf{configuration}
        \item \textbf{frontespizio}
        \item \textbf{diario delle modifiche}
        \item \textbf{informazioni}
                Questa pagina contiene le informazioni della riunione e l'ordine del giorno previsto.
                Le informazioni prevedono data, ora, luogo, lista partecipanti ed eventuali assenti.
        \item \textbf{svolgimento}
                L'equivalente del "contenuto" degli altri documenti, sarà un unico file e si 
                svilupperanno i punti scritti nell'ordine del giorno.
        \item \textbf{conclusioni-decisioni}
                La sezione conclusioni riassume in maniera sintetica quanto detto durante la riunione, 
                riassume ciò che è da fare nel breve futuro e se già stabilita si indica la data 
                della prossima riunione. \newline
                La tabella del tracciamento delle decisioni, costituita da due colonne, serve per 
                riportare in maniera schematica le decisioni prese e assegnare loro un codice 
                VI\_aaaa-mm-gg oppure VE\_aaaa-mm-gg in base a verbale interno o esterno, così 
                se necessario questo codice può essere usato per riferirsi alla decisione presa 
                e leggendo il verbale ne troverà la spiegazione.
\end{enumerate}
Il verbale è un documento che viene scritto, verificato e approvato una sola volta, 
non è soggetto al modo incrementale in quanto nel tempo non si aggiorna il verbale 
di una vecchia riunione e l'eventuale modifica di una decisione presa all'epoca verrà 
riportata solo nel nuovo verbale. Quindi i codici di versionamento sono sempre 0.0.1 
per l'amministratore nel ruolo di redattore, 0.1.0 superata la verifica e infine 1.0.0 
dopo che è stato approvato. Dopo l'approvazione il verbale verrà ufficialmente pubblicato 
e non si potrà più modificare.