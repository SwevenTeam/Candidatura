\newcommand{\docNome}{ Bozza motivazioni della scelta C1  }   % INSERIRE LA DATA IN FORMATO aaaa-mm-gg
\newcommand{\docVersione}{}                     % INSERIRE VERSIONE IN FORMATO x.y.z
\newcommand{\docStatus}{in lavorazione}         % AGGIORNARE SOLO QUANDO APPROVATO
\newcommand{\docUso}{}                          % INTERNO O ESTERNO
\newcommand{\docDestinatari}{Gruppo Sweven Team} %Se verbale esterno togliere questo comando (verranno inseriti direttamente come testo) 
\newcommand{\docNomeTeam}{Sweven Team}
\newcommand{\docRedattori}{}                    % REDATTORE
\newcommand{\docVerificatori}{}                 % VERIFICATORE
\newcommand{\docApprovazione}{}                 % APPROVATORE
\documentclass[12pt, a4paper,table]{article}
\usepackage[utf8]{inputenc}
\usepackage{fancyhdr}
\usepackage{geometry}
\usepackage{xcolor}
\usepackage{array}
\usepackage{graphicx}
\usepackage{hyperref}
\hypersetup{
pdfborder = {0 0 0}
}
\geometry{a4paper,top=3cm,bottom=3cm,left=2cm,right=2cm}
\title{\textsc{\docNome}}
\author{}
\date{}
\definecolor{footer-gray}{HTML}{808080}
\pagestyle{fancy}
\fancyhf{}
\rhead{\textcolor{footer-gray}{\docNome} }
\lhead{\textcolor{footer-gray}{Sweven Team}}
\fancyfoot{}
\cfoot{\textcolor{footer-gray}{Pagina \thepage } }
\renewcommand*\contentsname{Indice}
\begin{document}
	\maketitle
	\vspace{-3em}
	\begin{center}
	\includegraphics[scale=0.50]{images/logo.jpg} \\
	\vspace{2em}
	\huge \textsc{\docNomeTeam}\\
	\normalsize \href{mailto:swe7.team@gmail.com}{swe7.team@gmail.com}\\
	\vspace{2em}
	\end{center}
    \vspace{3em}

	\thispagestyle{empty}   
	\tableofcontents
	\newpage

	\section{Motivazioni della scelta}
	Dopo una discussione dei capitolati proposti e l'incontro con Imola informatica abbiamo deciso di scegliere il capitolato C1 e di seguito sono riportate le motivazioni:
	\begin{itemize}
		\item \textbf{Interesse comune per lo sviluppo di un chatbot}\\ Al momento della decisione, Bot4me è stata la prima scelta di tutti i membri del gruppo in quanto ci è sembrata l'idea più stimolante per noi e utile per l'azienda.
		\item \textbf{Nuove tecnologie}\\ Per il progetto è previsto l'uso di librerie e linguaggi che durante il percorso accademico non abbiamo affrontato. Questo capitolato, inoltre, rappresenta un'oppurtunità per arrichire le nostre conoscenze e di scoprire nuovi ambiti come quello del machine learning e lo sviluppo di applicazioni mobile.
		\item \textbf{Libertà sulle tecnologie da impiegare}\\ Nonostante l'azienda ci abbia già dato dei consigli su cosa utilizzare, lascia a noi la decisione finale.
		\item \textbf{Proponente aperto al confronto}\\ L'azienda si è dimostrata disponibile a chiarire eventuali dubbi e problemi fornendo, fin da subito, risorse informative e i propri contatti.
	\end{itemize}
\end{document}
	
	
