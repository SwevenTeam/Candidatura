\section{Capitolato C4}
		\begin{itemize}
			\item \textbf{Nome:} \textit{Guida Michelin social}: equivalente di Guida Michelin per post e storie;
			\item \textbf{Proponente}: \textit{Zero12};
			\item \textbf{Committenti}: \textit{Prof. Tullio Vardanega} e \textit{Prof. Riccardo Cardin}.
		\end{itemize}
		\subsection{Descrizione del capitolato}
		Il capitolato proposto dall'azienda \textbf{Zero12} offre la possibilità di realizzare una Guida Michelin rapportata però al mondo dei social con particolare interesse verso le piattaforme di \textbf{Instagram} e \textbf{TikTok}. Incronciando i dati dei post con le recensioni online, sarà possibile creare una mappa di posti suggeriti con la possibilità di specificare persone da seguire per la creazione della guida. La piattoforma fornirà inoltre la possibilità di monitorare le recensioni a partire da un luogo indicato. 
		\subsection{Finalità del progetto}
		Lo scopo di questo progetto è quello di sviluppare un'applicazione mobile \textit{iOS} e \textit{Android} in grado di mostrare la \textit{Guida Social Michelin}. Per realizzare questo obiettivo il team dovrebbe svolgere un'analisi sulle API di TikTok e Instagram al fine di indentificare il miglior approccio per la raccolta e l'analisi delle informazioni. Basandosi successivamente su un'architettura a microservizi per la creazione effettiva dell'applicativo. 
		\subsection{Tecnologie interessate}
		Per la realizzazione di questo capitolato, l'azienda proponente consiglia di utilizzare gli \textbf{Amazon Web Services}. In particolare facendo riferimento ai servizi di:
			\begin{itemize}
				\item \textbf{AWS Fargate}: servizio che permette la creazione di applicativi senza dover gestire i server
				\item \textbf{AWS Appsync}: servizio completamente gestito che facilita lo sviluppo di API GraphQL
				\item \textbf{Amazon Neptune}: servizio di database a grafo
			\end{itemize}
		Per quanto riguarda invece i linguaggi di programmazione, sarebbero necessari: \textbf{NodeJS}, \textbf{Swift} e \textbf{Kotlin}. Il primo utilizzato per sviluppare API Restful JSON, mentre i restanti verrebbero utilizzati rispettivamente per programmare applicazioni in ambito iOS e Android. 
		\subsection{Aspetti positivi}
		Il progetto proposto ha diversi aspetti positivi, in particolare la possibilità offerta dall'azienda \textbf{Zero12} di usufruire di attività di formazione sia sulle tecnologie AWS che su quelle Mobile che permetterebbe di coprire alcune lacune presenti su tali argomenti. Inoltre il gruppo ritiene che la possibilità di lavorare sull'elaborazione dei dati di due piattaforme social così importanti costituisca un ottimo bagaglio culturale da poter apprendere nel corso di questa offerta formativa. 
		\subsection{Criticità}
		Nello specifico il gruppo non ha rilevato particolari criticità relative allo sviluppo del progetto proposto. Le uniche incertezze emerse riguardano il livello di difficoltà nel riuscire a manipolare i dati offerti dalle piattoforme \textbf{Instagram} e \textbf{TikTok} al fine di ottenere il risultato voluto. Tuttavia non si ritiene essere un elemento limitante per lo sviluppo del progetto grazie anche al supporto eventuale offerto dall'azienda \textbf{Zero12}.
		\subsection{Valutazione finale}
		Nonostante il capitolato proposto risulti essere molto interessante offrendo la possibilità di lavorare con delle tecnologie che fornirebbero un bagaglio di cononscenze molto utile, a seguito di una votazione interna il team ha deciso di concentrare la propria candidatura su un altro progetto. 