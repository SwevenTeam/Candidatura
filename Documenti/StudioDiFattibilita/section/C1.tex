\section{Capitolato C1}
\begin{itemize}
	\item \textbf{Nome:} \textit{Bot 4 Me}: il mio nuovo collega digitale! ;
	\item \textbf{Proponente}: Imola Informatica ;
	\item \textbf{Committenti}: \textit{Prof. Tullio Vardanega} e \textit{Prof. Riccardo Cardin}.
\end{itemize}
\subsection{Descrizione del capitolato}
Il capitolato prevede la realizzazione di un chatbot, ovvero un "collega" digitale con il quale un utente, generalmente un dipendente dell'azienda, potrà interagire per svolgere diversi compiti in maniera automatica. 
\\Lo scopo principale del progetto consiste nel creare un tool dove dipendenti, neoassunti e non, possano trovare tutti gli strumenti aziendali necessari, aiutandoli inoltre nel loro utilizzo e semplificando alcune operazioni.
\\Viene richiesta la possibilità da parte dell'utente di compiere 2 attività fondamentali tramite l'utilizzo del chatbot :
\begin{itemize}
	\item consuntivare le attività giornaliere, ovvero registrare in un apposito database l'attività lavorativa in base alla tipologia e alle ore svolte;
	\item tracciare la propria presenza in sede.
\end{itemize}
\subsection{Finalità del progetto}
Il prodotto finale consisterà in un assistente personale, disponibile su diversi tipi di dispositivi, con cui l'utente potrà interagire fornendo lo specifico task e le informazioni appropriate, permettendo così al chatbot di svolgere il compito al posto dell'utente.
\\In modo più specifico, una singola interazione con il chatbot si svolgerà nel seguente modo :
\begin{itemize}
	\item tramite input, testuale o vocale, l'utente invia un comando al 
	chatbot. L'input verrà fornito in un linguaggio naturale per l'utente;
	\item l'input verrà inviato al server, il quale avrà il compito di interpretarlo e gestire la richiesta
	\item nel caso in cui siano necessari ulteriori dati, il chatbot dovrà richiederli all'utente;
	\item altrimenti, nel caso in cui i dati forniti siano sufficienti, verrà eseguita l'operazione lato server e restituita la condizione attuale della richiesta (se andata a buon fine o meno). 
\end{itemize}
Lato Front-end, Bot4me dovrà essere interagibile tramite PC e dispositivi mobile.
\\La GUI, oltre alla registrazione della presenza ed alla consuntivazione, potrà permettere diverse interazioni con l'utente, quali :
\begin{itemize}
	\item apertura cancelli tramite protocollo \textbf{MQTT}
	\item creazione ed apertura di nuove riunioni (\textbf{Google Meet o Zoom})
	\item ricerca di documenti tramite \textbf{CMIS}
	\item creazione di ticket su piattaforma \textbf{Redmine}
\end{itemize}

\subsection{Tecnologie interessate}
Riguardo le tecnologie utilizzabili, il proponente offre totale libertà, consigliando però alcune tecnologie in particolare, poiché più adatte rispetto ad altre :
\begin{itemize}
	\item \textbf{Python}: linguaggio di programmazione per la parte di back-end lato server, consigliato poiché presenta librerie come \textbf{Chatterbot}, utili alla realizzazione di un chatbot
	\item linguaggi per la realizzazione della parte di front-end che gestiscono l'interazione con l'utente : 
	\begin{itemize}
		\item \textbf{React}: per la versione pc, utilizza linguaggi di Mark Up (\textbf{HTML}/\textbf{CSS}) e di Programmazione (\textbf{Javascript}) per l'interazione con il chatbot tramite webApp
		\item \textbf{WebView React} per la versione mobile, ne viene consigliato l'utilizzo per adattare facilmente la webApp a dispositivi mobile
	\end{itemize}
	\item \textbf{API Rest}: usate per login tramite credenziali (\textbf{LDPA}) e interazioni con l'applicativo \textbf{EMT}, utilizzato per la consuntivazione, il tracciamento delle attività e la prenotazione delle postazioni in sede 
	\item \textbf{Azure Cloud} o \textbf{Amazon Web Services}: servizi utilizzati per la gestione lato server
\end{itemize}

\subsection{Aspetti positivi}
Diversi sono gli aspetti positivi riscontrati dal gruppo nel progetto.
\\In primo luogo, la \textbf{chiarezza espositiva} e la \textbf{disponibilità} fornita dal proponente, confermata dai vari consigli su quali tecnologie possano adattarsi meglio al progetto da svolgere. 
\\A seguire, la possibilità di interfacciarsi con \textbf{nuove tecnologie} con cui diversi membri del gruppo non sono ancora familiari,come per esempio il linguaggio di programmazione Python, che risulta essere sconosciuto ma anche molto interessante. 
\\Inoltre, la possibilità di interagire con programmi ed elementi interni ad una \textbf{realtà aziendale}, come l'applicativo EMT per le consuntivazioni. 
\\Infine, l'idea stessa di creare un assistente testuale ha
riscontrato un forte interesse all'interno del gruppo.
\subsection{Criticità}
La principale criticità consiste nell'utilizzo di tecnologie con cui buona parte del gruppo non ha ancora dimestichezza, come API Rest e il linguaggio Python.
\subsection{Valutazione finale}
Nel complesso, pensiamo che le criticità non siano insormontabili ed anzi, possano essere facilmente risolvibili e risultare in una ottima opportunità per migliorare ed approfondire diversi argomenti. 
\\Quindi il gruppo decide all'unanimità di scegliere questo capitolato, poiché ritenuto il più interessante e stimolante.