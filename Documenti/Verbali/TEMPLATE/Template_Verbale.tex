\newcommand{\docNome}{ Verbale Interno del  }   % INSERIRE LA DATA IN FORMATO aaaa-mm-gg
\newcommand{\docVersione}{}                     % INSERIRE VERSIONE IN FORMATO x.y.z
\newcommand{\docStatus}{in lavorazione}         % AGGIORNARE SOLO QUANDO CONFERMATO
\newcommand{\docUso}{}                          % INTERNO O ESTERNO
\newcommand{\docNomeProgetto}{}
\newcommand{\docNomeTeam}{Sweven Team}
\newcommand{\docRedattori}{}                    % REDATTORE
\newcommand{\docVerificatori}{}                 % VERIFICATORE
\newcommand{\docApprovazione}{}                 % APPROVATORE
\documentclass[12pt, a4paper,table]{article}
\usepackage[utf8]{inputenc}
\usepackage{fancyhdr}
\usepackage{geometry}
\usepackage{xcolor}
\usepackage{array}
\usepackage{graphicx}
\usepackage{hyperref}
\hypersetup{
pdfborder = {0 0 0}
}
\geometry{a4paper,top=3cm,bottom=3cm,left=2cm,right=2cm}
\title{\textsc{\docNome} \\ \textsc{\docNomeProgetto}}
\author{}
\date{}
\definecolor{footer-gray}{HTML}{808080}
\pagestyle{fancy}
\fancyhf{}
\rhead{\textcolor{footer-gray}{\docNome} }
\lhead{\textcolor{footer-gray}{Sweven Team}}
\fancyfoot{}
\cfoot{\textcolor{footer-gray}{Pagina \thepage } }
\renewcommand*\contentsname{Indice}
\begin{document}
	\maketitle
	\vspace{-3em}
	\begin{center}
	\includegraphics[scale=0.50]{images/logo.jpg} \\
	\vspace{2em}
	\huge \textsc{\docNomeTeam}\\
	\normalsize \href{mailto:swe7.team@gmail.com}{swe7.team@gmail.com}\\
	\vspace{2em}
	\begin{tabular}{r|l}
		\multicolumn{2}{c}{ \textsc{Informazioni sul documento} } \\
		\hline
		\textbf{Versione}     & \docVersione\\
		\textbf{Uso}          & \docUso\\
		\textbf{Stato}        & \docStatus\\
		\textbf{Redattori}    & \docRedattori\\
		\textbf{Verificatori} & \docVerificatori\\
		\textbf{Approvatori} & \docApprovazione\\
	\end{tabular}
	\end{center}
	\thispagestyle{empty}   
	\newpage
	
	\section*{Diario delle modifiche}
	\begin{center}
	\renewcommand{\arraystretch}{1.8} %aumento ampiezza righe
	\begin{tabular}{ |c|c|m{14em}|m{7em}|c| }
	\hline
	\textbf{Versione} & \textbf{Data} & \textbf{Descrizione} &  \textbf{Autore} &  \textbf{Ruolo} \\
	\hline
	1.0.0 & aaaa-mm-gg &  & \docApprovazione & Approvatore\\ % Se il nome approvatore non ci sta, metterlo a mano con aggiunta di \newline (esempio: Nome \newline Cognome)
	\hline
	0.1.0 & aaaa-mm-gg &  & \docVerificatori & Verificatore\\ % Stesso discorso di prima
	\hline
    0.0.1 & aaaa-mm-gg &  & \docRedattori & Redattore\\  % Stesso discorso di prima
	\hline
	\end{tabular}
	\end{center}
	\newpage
	
	\tableofcontents
	\newpage
	
	\section{Informazioni}
	\begin{itemize}
		\item \textbf{Data}:        % INSERIRE DATA
		\item \textbf{Orario}:      % INSERIRE ORA
		\item \textbf{Luogo}: Meeting Zoom
		\item \textbf{Partecipanti}:
		\begin{itemize}
			\item Irene Benetazzo
			\item Tommaso Berlaffa
			\item Mattia Episcopo
			\item Pietro Macrì
			\item Qi Fan Andrea Pan
			\item Matteo Pillon
			\item Samuele Rizzato
		\end{itemize}
		\item \textbf{Assenti}:
		Nessuno
		% SE QUALCUNO FOSSE ASSENTE, METTERLO QUI
		%\begin{itemize}
		    %\item "NomePersona"
		%\end{itemize}
	\end{itemize}
	\section{Ordine del giorno}
	%\begin{enumerate}
		%\item Titolo primo punto.
	%\end{enumerate}
	\newpage
	
	\section{Svolgimento}
		\subsection {} Prima subsection. Se altre, copia e incolla
	\section{Conclusioni}
	    \subsection {} Inserire qui le conclusioni
	\newpage
	
	\section*{Tracciamento delle decisioni}
	\renewcommand{\arraystretch}{1.8} % Aumento ampiezza righe
	\begin{tabular}{ |c|l| }
		\hline
		\textbf{Codice} & \textbf{Descrizione} \\
		\hline
		VI\_aaaa-mm-gg.1 & RIEMPIRE QUI SE VERBALE INTERNO\\ \hline
		VE\_aaaa-mm-gg.1 & RIEMPIRE QUI SE VERBALE ESTERNO\\ \hline
		% Inserire qui nuovi cambiamenti
	\end{tabular}
\end{document}