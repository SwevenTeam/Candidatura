\newcommand{\docNome}{ Verbale Interno del  2022-03-21}   % INSERIRE LA DATA IN FORMATO aaaa-mm-gg
\newcommand{\docVersione}{0.1.0}                     % INSERIRE VERSIONE IN FORMATO x.y.z
\newcommand{\docStatus}{In lavorazione}         % AGGIORNARE SOLO QUANDO CONFERMATO
\newcommand{\docUso}{Interno}                          % INTERNO O ESTERNO
\newcommand{\docNomeProgetto}{}
\newcommand{\docDestinatari}{Gruppo Sweven Team}
\newcommand{\docNomeTeam}{Sweven Team}
\newcommand{\docRedattori}{Matteo Pillon}                    % REDATTORE
\newcommand{\docVerificatori}{Qi Fan Andrea Pan}                 % VERIFICATORE
\newcommand{\docApprovazione}{Samuele Rizzato}                 % APPROVATORE
\documentclass[12pt, a4paper,table]{article}
\usepackage[utf8]{inputenc}
\usepackage{fancyhdr}
\usepackage{geometry}
\usepackage{xcolor}
\usepackage{array}
\usepackage{graphicx}
\usepackage{hyperref}
\hypersetup{
pdfborder = {0 0 0}
}
\geometry{a4paper,top=3cm,bottom=3cm,left=2cm,right=2cm}
\title{\textsc{\docNome} \\ \textsc{\docNomeProgetto}}
\author{}
\date{}
\definecolor{footer-gray}{HTML}{808080}
\pagestyle{fancy}
\fancyhf{}
\rhead{\textcolor{footer-gray}{\docNome} }
\lhead{\textcolor{footer-gray}{Sweven Team}}
\fancyfoot{}
\cfoot{\textcolor{footer-gray}{Pagina \thepage } }
\renewcommand*\contentsname{Indice}
\begin{document}
	\maketitle
	\vspace{-3em}
	\begin{center}
	\includegraphics[scale=0.50]{images/logo.jpg} \\
	\vspace{2em}
	\huge \textsc{\docNomeTeam}\\
	\normalsize \href{mailto:swe7.team@gmail.com}{swe7.team@gmail.com}\\
	\vspace{2em}
	\begin{tabular}{r|l}
		\multicolumn{2}{c}{ \textsc{Informazioni sul documento} } \\
		\hline
		\textbf{Versione}     & \docVersione\\
		\textbf{Uso}          & \docUso\\
		\textbf{Destinatari}  & \docDestinatari\\
		& Prof. Vardenega Tullio\\
		& Prof. Cardin Riccardo\\
		\textbf{Stato}        & \docStatus\\
		\textbf{Redattori}    & \docRedattori\\
		\textbf{Verificatori} & \docVerificatori\\
		\textbf{Approvatori} & \docApprovazione\\
	\end{tabular}
	\end{center}
	\vspace{3em}
    \begin{center}
        \LARGE{\textbf{Sintesi}} 
    \end{center}
    \normalsize{Incontro interno per la definizione di un template da utilizzare per redarre i documenti e per la discussione dei passi successivi da svolgere per la candidatura}
	\thispagestyle{empty}   
	\newpage
	
	\section*{Diario delle modifiche}
	\begin{center}
	\renewcommand{\arraystretch}{1.8} %aumento ampiezza righe
	\begin{tabular}{|c|c|m{14em}|m{7em}|c|}
	\hline
	\textbf{Versione} & \textbf{Data} & \textbf{Descrizione} &  \textbf{Autore} &  \textbf{Ruolo} \\
	\hline
	1.0.0 & aaaa-mm-gg & xyz & \docApprovazione & Approvatore\\ % Se il nome approvatore non ci sta, metterlo a mano con aggiunta di \newline (esempio: Nome \newline Cognome)
	\hline
	0.1.0 & 2022-03-25 & Validazione del documento & \docVerificatori & Verificatore\\ % Stesso discorso di prima
	\hline
    0.0.1 & 2022-03-21 &  Redazione del documento & \docRedattori & Redattore\\  % Stesso discorso di prima
	\hline
	\end{tabular}
	\end{center}
	\newpage
	
	\tableofcontents
	\newpage
	
	\section{Informazioni}
	\begin{itemize}
		\item \textbf{Data}: 2022-03-21        % INSERIRE DATA
		\item \textbf{Orario}: 18:10 - 18:50   % INSERIRE ORA
		\item \textbf{Luogo}: Meeting Zoom
		\item \textbf{Partecipanti}:
		\begin{itemize}
			\item Irene Benetazzo
			\item Tommaso Berlaffa
			\item Mattia Episcopo
			\item Pietro Macrì
			\item Qi Fan Andrea Pan
			\item Matteo Pillon
			\item Samuele Rizzato
		\end{itemize}
		\item \textbf{Assenti}:
		Nessuno
		% SE QUALCUNO FOSSE ASSENTE, METTERLO QUI
		%\begin{itemize}
		    %\item "NomePersona"
		%\end{itemize}
	\end{itemize}
	\section{Ordine del giorno}
	\begin{enumerate}
		\item Definizione template documenti
		\item Discussione attività svolgere per la candidatura
		\item Studio di fattibilità capitolati proposti
	\end{enumerate}
	\newpage
	
	\section{Svolgimento}
		\subsection {Definizione template documenti}
		In seguito alla creazione dei primi verbali il gruppo si è accorto della mancanza di uno standard con il quale creare i documenti necessari. Per rimediare a tale lacuna abbiamo deciso di realizzare un template da utilizzare per la redazione della documentazione. Di seguito vengono presentate le linee guida che sono state approvate: 
		\begin{itemize}
			\item Utilizzo della data in formato americano (\textit{aaaa-mm-gg})
			\item Aumento ampiezza righe all'interno delle tabelle per una migliore visualizzazione delle informazioni
			\item I file relativi ai verbali interni/esterni dovranno avere la stessa nomenclatura della cartella in cui sono contenuti
			\item Adempimento alle linee guida relative ai nomi dei ruoli coinvolti nella creazione dei documenti
		\end{itemize}
		\textbf{Irene Benetazzo} e \textbf{Mattia Episcopo} si assumono il compito di realizzare il template che segua le linee guida presentate. Una volta che sarà terminato i documenti fino ad ora prodotti saranno revisionati e aggiornati in modo da essere allineati tra di loro ad uno standard comune. 
		 
		 \subsection {Discussione attività svolgere per la candidatura}
		Al fine di preparare al meglio la candidatura per il capitolato scelto, il gruppo ritiene necessario che si esegua una fase di approffondimento individuale. Lo scopo di tale fase è quello di permettere ad ogni membro del team di sviluppare la propria idea relativa ai punti fondametali della candidatura, tra cui:
		\begin{itemize}
			\item Fornire una motivazione della scelta del capitolato
			\item Avanzare una proposta per la suddivisione delle ore lavorative
			\item Proporre uno stile per le slide che accompagneranno la candidatura
		\end{itemize}

		\newpage
		 \subsection {Studio di fattibilità capitolati proposti}
		In accordo comune il gruppo ha deciso di realizzare, in vista del prossimo incontro, l'analisi relativa allo studio di fattibilità dei capitolati proposti per il secondo lotto. Di seguito viene riportata una tabella che riassume i responsabili assegnati ai relativi progetti.  
		\vspace{1em} 
		
		 \renewcommand{\arraystretch}{1.8} % Aumento ampiezza righe
		 \begin{center}
		 \begin{tabular}{ |c|l| }
			\hline \multicolumn{2}{|c|}{\textbf{Studio di fattibilità }} \\
			\hline
			\textbf{Capitolato} & \textbf{Autore} \\
			\hline
			C1 - \textit{Bot4Me} & Pietro Macrì\\
			\hline
			C3 - \textit{CC4D} & Samuele Rizzato\\
			\hline
			C4 - \textit{Guida Michelin social }& Matteo Pillon\\
			\hline
		 \end{tabular}
		\end{center}
	
		 \section{Conclusioni}
	     In seguito all'incontro il gruppo approva la definizione di un template da utilizzare per la stesura dei documenti. Sono stati inoltre definiti dei compiti da svolgere prima del prossimo meeting, fissato per il giorno 2022-03-28 alle ore 18:10. 
	\newpage
	
	\section*{Tracciamento delle decisioni}
	\renewcommand{\arraystretch}{1.8} % Aumento ampiezza righe
	\begin{tabular}{ |c|l| }
		\hline
		\textbf{Codice} & \textbf{Descrizione} \\
		\hline
		VI\_2022-03-21.1 & Approvazione template per redarre i documenti \\ \hline
		% Inserire qui nuovi cambiamenti
	\end{tabular}
\end{document}
\end{document}