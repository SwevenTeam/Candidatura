\newcommand{\docNome}{ Verbale Interno del 2022-03-14 }
\newcommand{\docVersione}{0.0.1}
\newcommand{\docStatus}{in lavorazione}
\newcommand{\docUso}{interno}
\newcommand{\docNomeProgetto}{}
\newcommand{\docNomeTeam}{Sweven Team}
\newcommand{\docRedattori}{Mattia Episcopo}
\newcommand{\docVerificatori}{Pietro Macrì}
\newcommand{\docApprovazione}{Irene Benettazzo}
\documentclass[12pt, a4paper,table]{article}
\usepackage[utf8]{inputenc}
\usepackage{fancyhdr}
\usepackage{xcolor}
\usepackage{array}
\usepackage{graphicx}
\usepackage{hyperref}
\hypersetup{
pdfborder = {0 0 0}
}
\title{\textsc{\docNome} \\ \textsc{\docNomeProgetto}}
\author{}
\date{}
\definecolor{footer-gray}{HTML}{808080}
\pagestyle{fancy}
\fancyhf{}
\rhead{\textcolor{footer-gray}{\docNome} }
\lhead{\textcolor{footer-gray}{Sweven Team}}
\fancyfoot{}
\cfoot{\textcolor{footer-gray}{Pagina \thepage } }
\renewcommand*\contentsname{Indice}
\begin{document}
	\maketitle
	\vspace{-3em}
	\begin{center}
	\includegraphics[scale=0.50]{images/logo.jpg} \\
	\vspace{2em}
	\huge \textsc{\docNomeTeam}\\
	\normalsize \href{mailto:swe7.team@gmail.com}{swe7.team@gmail.com}\\
	\vspace{2em}
	\begin{tabular}{r|l}
		\multicolumn{2}{c}{ \textsc{Informazioni sul documento} } \\
		\hline
		\textbf{Versione}     & \docVersione\\
		\textbf{Uso}          & \docUso\\
		\textbf{Stato}        & \docStatus\\
		\textbf{Redattori}    & \docRedattori\\
		\textbf{Verificatori} & \docVerificatori\\
		\textbf{Approvazione} & \docApprovazione\\
	\end{tabular}
	\end{center}
	\thispagestyle{empty}   
	\newpage
	\section*{Diario delle modifiche}
	\begin{center}
	\begin{tabular}{ |c|c|m{9em}|m{5em}|c| }
	\hline
	\textbf{Versione} & \textbf{Data} & \textbf{Descrizione} &  \textbf{Autore} &  \textbf{Ruolo} \\
	\hline
	0.0.1 & 2022-03-15 & Redazione del documento & Mattia Episcopo & Redattore\\
	\hline
	\end{tabular}
	\end{center}
	\newpage
	\tableofcontents
	\newpage
	\section{Informazioni}
	\begin{itemize}
		\item \textbf{Data}: 2022-03-14
		\item \textbf{Orario}: 18:15 - 19:15
		\item \textbf{Luogo}: Meeting Zoom
		\item \textbf{Partecipanti}:
		\begin{itemize}
			\item Irene Benettazzo
			\item Tommaso Berlaffa
			\item Mattia Episcopo
			\item Pietro Macrì
			\item Qi Fan Andrea Pan
			\item Matteo Pillon
			\item Samuele Rizzato
		\end{itemize}
	\end{itemize}
	\section{Ordine del giorno}
	\begin{enumerate}
		\item Decisioni riguardo l'identità del gruppo
		\item Discussione e scelta del capitolato
		\item Organizzazione di comunicazione e condivisione materiale all'interno del gruppo
		\item Informazioni da richiedere sul capitolato scelto
	\end{enumerate}
	\newpage
	\section{Svolgimento}
		\subsection{Decisioni riguardo l'identità del gruppo}
		Dopo una breve discussione dove i membri del gruppo hanno espresso idee e preferenze è stato scelto come nome del gruppo Sweven Team, come proposto da Mattia Episcopo. In seguito si sono valutate varie proposte per il logo del gruppo, e dopo averle valutate con attenzione si è scelto quello proposto da Matte Pillon. Come richiesto dal regolamento del progetto didattico, il gruppo ha poi attivato la propria email: swe7.team@gmail.com che verrà usata per tutte le comunicazioni. Infine il gruppo ha scelto di tenere il girono lunedì e l'ora 18:00 come incontro con cadenza settimanale per le prossime riunioni.
		\subsection{Discussione e scelta del capitolato}
		Sono stati valutati insieme i capitolati a disposizione. Dopo una discussione riguardo la fattibilità dei capitolati, il gruppo si è orientato verso la scelta del capitolato C1 e proprio per approfondire questa scelta richiederà un incontro con l'azienda proponente. Si è quindi tenuta una breve sessione in cui i membri del gruppo hanno raggruppato idee e domande per l'incontro con l'azienda proponente.
		\subsection{Organizzazione di comunicazione e condivisione materiale all'interno del gruppo}
		Il gruppo ha deciso all'unanimità di utilizzare GitHub per la condivisione e il versionamento dei documenti che riguardano il progetto e l'organizzazione interna. Inoltre per la stesura dei documenti del gruppo si è deciso di utilizzare \LaTeX .
		\subsection{Informazioni da richiedere sul capitolato scelto}
		Chiarimento sui vari protocolli da utilizzare per le funzionalità richieste dall'applicativo.
		Linguaggi di programmazione da utilizzare per lo sviluppo dell'applicativo, in particolare per la realizzazione della parte mobile.
	\section{Conclusioni}
	Si richiede all'azienda proponente un incontro per la settimana corrente allo scopo di chiarire alcuni punti sul capitolato scelto.
	\newpage
	\section*{Tracciamento delle decisioni}
	\begin{tabular}{ |c|l| }
		\hline
		\textbf{Codice} & \textbf{Descrizione} \\
		\hline
		VI\_2022-03-14.1 & Scelto nome del gruppo: Sweven Team \\ \hline
		VI\_2022-03-14.2 & Approvato logo del gruppo\\ \hline
		VI\_2022-03-14.3 & Attivata email del gruppo: swe7.team@gmail.com\\ \hline
		VI\_2022-03-14.4 & Stabilite riunioni a cadenza settimanale: Lunedì 18:00\\ \hline
		VI\_2022-03-14.5 & Scelto sistema di versionamento: GitHub\\ \hline
		VI\_2022-03-14.6 & Scelto strumento per la stesura dei documenti: \LaTeX\\ \hline
		VI\_2022-03-14.7 & Scelta capitolato: C1 - Bot4Me\\
		\hline
	\end{tabular}
\end{document}