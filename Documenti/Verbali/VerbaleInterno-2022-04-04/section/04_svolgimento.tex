    \section{Svolgimento}

    \subsection {Incontro dell'8 aprile}
    Il gruppo ha discusso circa l'evento del giorno 8 aprile. L'inizio del dialogo è avvenuto con l'esposizione delle slide precedentemente preparate dal collega \textbf{Matteo Pillon}.
    \newline
    Alla presentazione è seguita la suddivisione dei ruoli che dovranno tenere le persone le quali hanno precedentemente dato disponibilità alla partecipazione, presso la sede dell'università, alla giornata.
    Da un primo approccio, basato principalmente sulla suddivisione in base alla tematica trattata, si è virati poi verso un altro tipo di approccio, improntato più sulle tempistiche mostrate in un comunicato sulla piattaforma Moodle.
    \newline
    Si è giunti al seguente risultato:
    \renewcommand{\arraystretch}{1.8} % Aumento ampiezza righe
    \begin{center}
    \begin{tabular}{ |c|l| }
        \hline \multicolumn{2}{|c|}{\textbf{Suddivisione temi}} \\
        \hline
        \textbf{Tema} & \textbf{Relatore} 
        \\
        \hline
        Tecnologie per il progetto & \textit{Matteo Pillon}
        \\
        \hline
        Motivazioni della scelta & \textit{Tommaso Berlaffa}
        \\
        \hline
        Costi e preventivo & \textit{Pietro Macrì} oppure$^{(1)}$ \textit{Irene Benetazzo}\\
        \hline
    \end{tabular}
    \newline
    \end{center}
     $^{(1)}$La partecipazione di Irene dipenderà dalle sue condizioni di salute

    \subsection {Stato dei documenti redatti}
    Il gruppo ha analizzato la situazione concernente lo stato dei documenti redatti finora. 
    \\ Da questa discussione è emersa l'esigenza
    di verbalizzazione e approvazione di alcuni verbali riguardanti altri precedenti incontri settimanali.
    \newline
    I partecipanti si sono dunque suddivisi gli ultimi compiti inerenti e hanno fissato una scadenza di circa 24 ore, al
    fine di spostare più documenti possibile nella sezione di quelli completati e organizzare quindi con maggiore efficienza
    la sezione di sviluppo presente al momento sulla piattaforma GitHub.

    \vspace{1em} 

    