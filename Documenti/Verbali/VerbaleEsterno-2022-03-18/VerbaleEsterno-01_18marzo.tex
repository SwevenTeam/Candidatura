\newcommand{\docNome}{ Verbale Esterno del 18-03-2022 }
\newcommand{\docVersione}{0.0.1}
\newcommand{\docUso}{Esterno}
\newcommand{\docDestinatari}{Gruppo Sweven Team}
\newcommand{\docStatus}{in lavorazione}
\newcommand{\docNomeProgetto}{} % ??
\newcommand{\docNomeTeam}{Sweven Team}
\newcommand{\docRedattori}{Irene Benetazzo}
\newcommand{\docVerificatori}{Mattia Episcopo}
\newcommand{\docApprovazione}{Pietro Macrì}
\documentclass[12pt, a4paper,table]{article}
\usepackage[utf8]{inputenc}
\usepackage{geometry}
\usepackage{fancyhdr}
\usepackage{xcolor}
\usepackage{array}
\usepackage{graphicx}
\usepackage{hyperref}
\hypersetup{pdfborder = {0 0 0}}
\geometry{a4paper,top=3cm,bottom=3cm,left=2cm,right=2cm}
\title{\textsc{\docNome} \\ \textsc{\docNomeProgetto}}
\author{}
\date{}
\definecolor{footer-gray}{HTML}{808080}
\pagestyle{fancy}
\fancyhf{}
\rhead{\textcolor{footer-gray}{\docNome} }
\lhead{\textcolor{footer-gray}{Sweven Team}}
\fancyfoot{}
\cfoot{\textcolor{footer-gray}{Pagina \thepage } }
\renewcommand*\contentsname{Indice}

\begin{document}
	\maketitle
	\vspace{-3em}
	\begin{center}
	\includegraphics[scale=0.50]{images/logo.jpg} \\
	\vspace{2em}
	\huge \textsc{\docNomeTeam}\\
	\normalsize \href{mailto:swe7.team@gmail.com}{swe7.team@gmail.com}\\
	\vspace{2em}
	\begin{tabular}{r|l}
		\multicolumn{2}{c}{ \textsc{Informazioni sul documento} } \\
		\hline
		\textbf{Versione}     & \docVersione\\
		\textbf{Uso}          & \docUso\\
        \textbf{Destinatari}  & \docDestinatari\\
		\textbf{Stato}        & \docStatus\\
		\textbf{Redattori}    & \docRedattori\\
		\textbf{Verificatori} & \docVerificatori\\
		\textbf{Approvatori} & \docApprovazione\\
	\end{tabular}
    \end{center}
    \vspace{3em}
    \begin{center}
        \LARGE{\textbf{Sintesi}} 
    \end{center}
    \normalsize{Incontro informativo sul capitolato C1 proposto dall'azienda Imola Informatica}
    \thispagestyle{empty}   
	\newpage

	\section*{Diario delle modifiche}
	\begin{center}
	\renewcommand{\arraystretch}{1.8} %aumento ampiezza righe
	\begin{tabular}{ |c|c|m{14em}|m{7em}|c| }
	\hline
	\textbf{Versione} & \textbf{Data} & \textbf{Descrizione} &  \textbf{Autore} &  \textbf{Ruolo} \\
	\hline
    & 2022-03-XX & & & \\
	\hline
    & 2022-03-XX & & & \\
	\hline
	0.0.1 & 2022-03-19 & Redazione del documento & Irene \newline Benetazzo & Redattore\\
	\hline
	\end{tabular}
	\end{center}
	\newpage

	\tableofcontents
	\newpage

	\section{Informazioni}
	\begin{itemize}
		\item \textbf{Data}: 2022-03-18
		\item \textbf{Orario}: 11:00 - 11:45
		\item \textbf{Luogo}: Meeting Zoom
		\item \textbf{Partecipanti}:
		\begin{itemize}
			\item Irene Benetazzo
			\item Tommaso Berlaffa
			\item Mattia Episcopo
			\item Pietro Macrì
			\item Qi Fan Andrea Pan
			\item Matteo Pillon
			\item Samuele Rizzato
			\item Lorenzo Giacomo (Imola)
            \item Patera Lorenzo (Imola)
            \item Proscia Alessandro (Imola)            
		\end{itemize}
        \item \textbf{Assenti}: Nessuno
	\end{itemize}
	\section{Ordine del giorno}
	\begin{enumerate}
		\item L'applicazione in generale
		\item Scelte tecnologiche per sviluppare l'applicazione
		\item Sistema di login aziendale LDPA
		\item Informazioni sull'applicazione web aziendale EMT
		\item Supporto e disponibilità dell'azienda
		\item Confronto all'interno del gruppo
	\end{enumerate}
	\newpage

	\section{Svolgimento}
		\subsection{L'applicazione in generale}
		Il gruppo inizia chiedendo se l'azienda si attende un'applicazione principalmente 
		per pc o per dispositivi mobili, e se per il lato client si desidera un app nativa 
		o se è possibile interfacciarsi con le comuni app di messaggistica. \newline
		L'azienda spiega che la richiesta è di una Web App e poi consiglia l'aggiunta di una 
		web view per adattarla facilmente anche ai dispositivi mobili. Il lavoro è principalmente 
		nel lato server dell'applicazione; il lato client non è espressamente richiesto e si è 
		completamente liberi di creare una propria interfaccia o di intefacciarsi con un app di 
		messaggistica. Ad esempio interagire tramite le API con Telegram che offre già tutte 
		le varie versioni desktop e mobile.\newline
        Viene posta la domanda riguardante l'input, l'azienda comunica che l'input vocale 
		è facoltativo, mentre per l'input testuale si immagina perlopiù un botta e risposta, 
		da considerare anche il caso di un unico messaggio completo di tutte le informazioni necessarie.\newline
		Il gruppo chiede riguardo le richieste opzionali se ci sono preferenze, l'azienda invita 
		prima a partire dalle richieste obbligatorie, in particolare la parte di consuntivazione, 
		e poi per le richieste opzionali non hanno preferenze.
		L'azienda invita a realizzare un'applicazione in cui la struttura e l'interazione siano
		separate e si interagisca con dei sistemi esterni.
		
        \subsection{Scelte tecnologiche per sviluppare l'applicazione}
        L'azienda lascia completamente libero il gruppo di scegliere le tecnologie, tuttavia 
        consigliano Phyton lato server per le numerose librerie che offre a riguardo, 
		ad esempio \href{https://chatterbot.readthedocs.io/en/stable/}{ChatterBot}

		\subsection{Sistema di login aziendale LDPA}
		I referenti di Imola spiegano al gruppo che il sistema di login aziendale è strutturato ad albero 
		per unità organizzative, all'interno ci sono i gruppi e poi gli utenti a cui sono stati assegnati degli attributi. 
		Per la sicurezza non si fa mai un collegamento diretto e non si richiedono mai direttamente le credenziali
		ma vengono esposte una serie di APIRest e verranno autenticate tramite un token. 
		Cioè tramite API si richiedono delle credenziali da usare per il ChatBot e come risposta si avrà 
		una stringa contenente tutte le informazioni. Suggeriscono il sito 
		\href{https://www.redhat.com/it/topics/api/what-is-a-rest-api}{https://www.redhat.com/it/topics/api/what-is-a-rest-api},
		comunque maggiori informazioni verranno fornite dopo l'assegnazione dell'appalto.
		
		\subsection{Informazioni sull'applicazione web aziendale EMT}
		L'azienda mostra a schermo l'applicativo EMT attualmente in uso facendo vedere l'esempio di form 
		da compilare per poter consuntivare le ore di lavoro. Quindi tramite il ChatBot lo scopo è riuscire 
		ad ottenere le informazioni necessarie, validando i vari messaggi e in caso di ambiguità va chiesta
		chiarezza all'utente.

		\subsection{Supporto e disponibilità dell'azienda}
		Per eseguire i test l'azienda invita ad usare le varie convenzioni disponibili per studenti 
		(ad esempio il pacchetto studenti di GitHub), comunuque in caso di difficoltà o necessità 
		l'azienda è disposta a collaborare per trovare una soluzione.
		I tre referenti presenti, si mostrano fin da subito molto disponibili e forniscono
		anche i loro contatti Telegram. \newline
		Suggeriscono durante la fase di progettazione e sviluppo di mantenere il più possibile 
		la semplicità, inoltre sottolineano che per il gruppo è in primis un'esperienza formativa, 
		quindi invitano a non esitare a contattarli in caso di dubbi e difficoltà. 
		Infine consigliano al gruppo di partire da una PoC basata su tecnologie classiche.
		
		\subsection{Confronto all'interno del gruppo}
		Alle 11.35 termina la riunione con l'azienda Imola e il gruppo Sweven rimane in chiamata Zoom
		per confrontarsi sulle informazioni appena ricevute. \newline
		Tutti i membri del gruppo si reputano soddisfatti del colloquio per le delucidazioni ricevute 
		che hanno sciolto le domande sorte durante la lettura del pdf di presentazione del capitolato:
		in particolare non si era colto che la richiesta principale è il lato server dell'applicazione.
		L'azienda lascia si molta libertà, che può essere un'arma a doppio taglio, ma comunque hanno 
		fornito suggerimenti come l'utilizzo il linguaggio Phyton, di usare le APIRest per interagire.
		Inoltre il gruppo è rimasto colpito positivamente dall'azienda avendo un'ottima impressione e
		in particolare dalla disponibilità mostrata.
	
	\section{Conclusioni}
	Il gruppo approva all'unanimità la scelta del capitolato C1 con l'azienda Imola Informatica.
	\newpage
	
	\section*{Tracciamento delle decisioni}
	\renewcommand{\arraystretch}{2} %aumento ampiezza righe
	\begin{tabular}{ |m{8em}|m{30em}| }
		\hline
		\textbf{Codice} & \textbf{Descrizione} \\
		\hline
		VE\_2022-03-14.1 & Approvazione del capitolato C1 \\
		\hline
	\end{tabular}
\end{document}